\documentclass{article}
\usepackage[utf8]{inputenc}
\usepackage[spanish]{babel}
\usepackage{listings}
\usepackage{graphicx}
\graphicspath{ {images/} }
\usepackage{cite}

\begin{document}

\begin{titlepage}
    \begin{center}
        \vspace*{1cm}
            
        \Huge
        \textbf{Parcial 1 - Calistenia}
            
        \vspace{0.5cm}
        \LARGE
        Informática II
            
        \vspace{1.5cm}
            
        \textbf{Valentina Ciro Medellín}
            
        \vfill
            
        \vspace{0.8cm}
            
        \Large
        Despartamento de Ingeniería Electrónica y Telecomunicaciones\\
        Universidad de Antioquia\\
        Medellín\\
        Marzo de 2021
            
    \end{center}
\end{titlepage}


\newpage 

A continuación se documenta la solución al desafío de describir cómo llevar dos objetos de una posición A a una posición B.
\vspace{0.8cm}
\section*{INSTRUCCIONES}
Por favor siga los siguientes pasos:
\vspace{0.8cm}

\textbf{1.} Escoja una sola mano para realizar todo el ejercicio.
\vspace{0.5cm}


\textbf{2.}Deslice la hoja de papel con la mano escogida para obtener dos tarjetas.
\vspace{0.5cm}

\textbf{3.}Tome las dos tarjetas con la misma mano.

\vspace{0.5cm}

\textbf{4.}Junte las tarjetas de tal forma que queden pegadas.
\vspace{0.5cm}

\textbf{5.}Ubique las tarjetas paradas de forma vertical en el centro de la hoja.
\vspace{0.5cm}

\textbf{6.} Con la misma mano, sujete las tarjetas de la superficie superior ayudándose del dedo índice y/o medio.
\vspace{0.5cm}

\textbf{7.} Con las tarjetas sujetadas en la parte superior, intente separar las tarjetas en la parte inferior ayudándose con los demás dedos sin despegarlas de arriba. 
\vspace{0.5cm}

\textbf{8.} Cuando obtenga una especie de pirámide, suelte las tarjetas cuidadosamente evitando que éstas se caigan para que mantengan su posición de pirámide.


\end{document}
